\section{Analysis of Cooperation Mechanisms}

\subsection{Overview}

\subsection{Contract Net}

\subsubsection*{General Description}

The Contract Net protocol was developed by R.G.Smith in 1980 \cite{Smith:1980:CNP:1311969.1312903} and is based on the way in which companies put contracts out to tender. There are five phases the protocol runs through to resolve a certain task:

\begin{itemize}
	\item \textbf{Recognition}: An agent (called the ``manager'') recognizes a task that has to be done and that it cannot solve or does not want to solve.	
	\item \textbf{Announcement}: The manager announces the task to all agents (general broadcast), a group of agents (limited broadcast), or specific agents (point-to-point announcement). The specification of the task includes a description of the task, constraints, and meta-information.
	\item \textbf{Bidding}: When an agent receives a task evaluate the task with respect to their specific properties and, if they are able and willing to solve the task, they submit a tender to the manager, indicating the relevant capabilities of the agent to solve the task.
	\item \textbf{Awarding}: A manager may receive several tenders for a task. Based on the information provided in the tenders, it chooses the best fitting agent or agents to execute the task. Chosen agents are called contractors for this task.
	\item \textbf{Expediting}: The chosen contractor or contractors then execute the task. If there is more than one contractor, further cooperation mechanisms might be necessary.
\end{itemize}

The Contract Net is based on a mutual selection process: Managers can select the best fitting tender and contractors can decide, for which announced tasks they want to create a tender.  

\subsubsection*{Application to Practical Work}

\subsection{Auctions}

\subsection{Coalitions}

\subsection{PGP}

\subsection{Voting}

